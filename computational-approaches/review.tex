\documentclass[journal=jacsat,manuscript=article]{achemso}

%\usepackage{...}

\bibliographystyle{achemso}

\author{Zhen-Hao Xu, Krystel El Hage and Markus Meuwly}
\affiliation{Department of Chemistry, University of Basel, Klingelbergstrasse 80, CH-4056 Basel, Switzerland.}
\email{m.meuwly@unibas.ch}

\date{\today}

\title{}

\begin{document}

\begin{abstract}

\end{abstract}

\clearpage

\section{Explicit Proton Transfer: The MMPT force field}
Proton transfer (PT) reactions are fundamental in biophysical and
biochemical processes. In order to characterize the properties of a
shared proton between an acceptor (A) and donor (D) moiety various
experimental methods have been used in the past. One of the most
successful approaches is based on optical
spectroscopy.\cite{wolke.sci.2016.watercluster,fournier.pnas.2014.watercluster,mackeprangmeuwly.pccp.2016.mmpt,wolkejohnson.jpca.2015.oxalate,howardmeuwly.jpca.2015.mmpt} 


\noindent
Molecular Mechanics with Proton Transfer (MMPT) is a force field-based
method which allows bond formation and bond breaking between the
transferring hydrogen atom and the acceptor or donor atom,
respectively.\cite{lammersmeuwly.jcc.2008.mmpt} In this approach,
multi-dimensional potential energy surfaces (PESs) are parametrized
from \emph{ab initio} calculations and fit to efficient
representations based on Morse potentials. The additional MMPT-energy
is written as
\begin{equation}
V_{\textrm{MMPT}}=\left\{
\begin{array}{ll}
V_{0}(R,\rho)+k\cdot\theta^2\textrm{, if a proton moves in a linear path}\\
V_{0}(R,\rho)+V_{d}(R,\rho,d)\textrm{, if a proton moves in a non-linear path}
\end{array}
\right.
\label{eq:vmmpt}
\end{equation}
where $R$ is the distance of D and A and $r$ is the distance of D and
H*. These two variables are combined in a progression coordinate
$\rho$ defined as $\rho=(r-r_0)/(R-R_0)\in [0,1]$ with $r_0 = 0.8$
\AA\/ and $R_0 = 1.6$ \AA\/. In Eq. \ref{eq:vmmpt}, the 2D potential
$V_{0}(R,\rho)$ is in a form of superposed Morse potential functions,
which associates with the major coordinates regarding the PT
reactions. In a linear PT reaction, a third dimension, $\theta$,
involves the angle $\angle D-\textrm{H*}-A$ and is approximated by
harmonic
function.\cite{lammersmeuwly.jcc.2008.mmpt,lammersmeuwly.jpca.2007.mmptdpt}
To extend the applicability of the MMPT force field, Yang \emph{et
  al.}\cite{yangmeuwly.jcp.2010.mmpt} further developed this force
field for describing reaction systems with non-linear hydrogen bonds.
Unlike the linear PT reaction, the MMPT potential in a non-linear form
regards the third dimension as $d = r \cdot \sin \theta$ and
$V_{d}(R,\rho,d)$ represents for a specific harmonic constrain. In
this study, the parametrization of MMPT force field has been validated
from the infrared spectra of the experiments.  \\



\subsection{Computational Infrared Spectroscopy for H-bonding Systems}

\noindent
In an attempt to assign infrared spectroscopic features, Howard
\emph{et al.}  investigated the infrared and near-infrared spectra
(NIR) of acetylacetone (AcAc), acetylacetone-$ d_8 $ and
hexafluoroacetylacetone\cite{howardmeuwly.jpca.2015.mmpt}.  In this
study, the fundamental OH-stretching bands were reported and
red-shifted relative to those of usual OH stretching transitions. By
using MMPT force field, the computed spectra from atomistic
simulations of AcAc in the gas phase reproduce most of experimental
records, especially the OH-stretching (or proton transfer) band was
found broad and week. This band is located in the region of 2000 cm$
^{-1} $ $ \sim $ 3000 cm$^{-1}$ which is in good agreement with
experimental observation. Furthermore, the location of this band was
found that has a sensitive dependence of the barrier height for PT
reaction and a barrier of 2.4 kcal/mol was concluded.  \\


\begin{figure}
\includegraphics[width=\textwidth]{figs/pes-2d.png}
\caption{Mixed two-dimensional PESs for double proton transfer in
  formic acid dimer. The reference data from MP2 calculations are in
  red and the empirical PES in black. The right hand panel illustrates
  that the empirical surface is of very high quality.}
\label{fig:pes-dpt}
\end{figure}


\noindent
Regarding the empirical relation between the position of OH-stretching
band and the barrier height of PT reactions, this is also an issue of
how to obtain spectroscopically accurate force field for atomistic
simulations. Such an interest was also concerned in a recent study for
double proton transfer (DPT) in formic acid dimer
(FAD)\cite{mackeprangmeuwly.pccp.2016.mmpt}. In addition to the
conventional MMPT force field, a coupling effect was introduced since
the DPT reaction of FAD occurs in a concerted pathway.  In MMPT force
field, the total potential energy of a DPT reaction can be given as
\begin{equation}
V_{\rm DPT} (r_1,R_1,\theta_1,r_2,R_2,\theta_2)=\big[V_{\rm 
sym}(r_1,R_1,\theta_1)\cdot\gamma+V_{\rm asy}(r_1,R_1,\theta_1)\cdot(1-\gamma)\big]
\label{eq:edpt}
\end{equation}
{\centering$ +\big[V_{\rm sym}(r_2,R_2,\theta_2)\cdot\gamma+V_{\rm asy}(r_2,R_2,\theta_2)\cdot(1-\gamma)\big], $}
\\
\\
\noindent
where $ \{r_1,R_1,\theta_1\} $ and $ \{r_2,R_2,\theta_2\} $ stand for
the coordinates of two PT moieties and the coupling is parametrized as
a switch factor, $\gamma$:
\begin{equation}
\gamma(r_1,r_2,R_1,R_2)=\frac{1}{2}\big\{1+\tanh[\sigma\cdot(r_1\cdot
  R_1-R_1^2/2)]\cdot\tanh[\sigma\cdot(r_2\cdot R_2-R_2^2/2)]\big\}
\label{eq:gamma}
\end{equation}
\noindent
Such a combination of symmetric double minimum (SDM) and symmetric
single minimum (SSM) PESs is shown in Figure \ref{fig:pes-dpt}. In a
next step the barrier height for the proton transfer needs to be
adjusted. This is accomplished through morphing the PES to reproduce
either a computed target barrier height or it is adjusted until the
observable of interest is correctly described, see Figure
\ref{fig:pes-morph}.
\begin{equation}
V_{\rm DPT}^{\rm target}=\frac{\Delta E_b^{\rm target}}{\Delta E_b}\cdot V_{\rm DPT}
\label{eq:morph}
\end{equation}
Finally, a selection of DPT barriers were analyzed and a barrier
between 5 and 7 kcal/mol was inferred and compared with a barrier of
7.9 kcal/mol from quantum calculations at the level of
CCSD(T)/aug-cc-pVTZ.  Therefore, combining experiment and atomistic
simulations allowed us to estimate the barrier for proton transfer
through comparison of spectroscopic signatures in the OH-stretching
regions.


\begin{figure}
\includegraphics[width=\textwidth]{figs/pes-morph.png}
\caption{The empirical correlation between morphed potential energy surface and bond stretching frequencies.}
	\label{fig:pes-morph}
\end{figure}


\subsection{Kinetic Isotope Effect from Path Integral Simulations Using the MMPT Potential}
Malonaldehyde (MA) has long served as a typical hydrogen transfer
system to test and validate various computational
approaches. Experimentally, the ground state tunneling splitting was
determined to be 21.58314 cm$^{-1}$ by different experiments with very
high accuracy.~\cite{mawilson81,mafirth91} Infrared spectra of MA have
also been recorded at high
resolution.~\cite{maexp83,maexp89,maexp92,maexp04}\\

\noindent
MMPT simulations with a generalization of the method to nonlinear
H-bonds were carried out in order to determine the tunneling
splittings for H- and D-transfer and to locate the position of the
proton-transfer band in the infrared.\cite{MM10ma} Building on a
harmonic bath averaged Hamiltonian (HBA) the effective reduced mass
was chosen such as to reproduce the tunneling splitting for
H-transfer. The effective reduced mass differs from the mass of the
transferring hydrogen atom due to kinetic coupling in the
system.\cite{MM10ma} However, the effective mass of the deuterated
species is then determined by usual isotopic mass ratios which allows
to validate the model because no new parameters are required. The
computed tunneling splittings of 22.0 cm$^{-1}$ and 2.9 cm$^{-1}$
compare favourably with the experimentally determined ones which are
21.583 cm$^{-1}$ and 2.915 cm$^{-1}$,
respectively.\cite{mawilson81,mafirth91,wilson84:2260} The proton
transfer mode exhibits a large red shifts relative to usual
OH-stretching vibrations and is found at 1543 cm$^{-1}$.\\

\noindent
Building on this MMPT potential a quantum mechanical treatment of the
kinetic isotope effect (KIE) in MA was attempted. The KIE relates the
rate constants for hydrogen and deuterium transfer via
$\mathrm{KIE}=k_{\mathrm{H}}/k_{\mathrm{D}}$. The KIE for the
intramolecular hydrogen transfer in MA has not been determined
experimentally. Combining a fully dimensional and validated
PES~\cite{MM10ma} based on molecular mechanics with proton transfer
(MMPT)~\cite{lammersmeuwly.jcc.2008.mmpt} with quantum instanton
(QI)\cite{miller03,voth10ma} path integral Monte Carlo (PIMC)
simulation the primary H/D KIE on the intramolecular proton transfer
in MA was found to be $5.2 \pm 0.4$ at room temperature.\cite{MM14ma}
For higher temperatures the KIE tends to 1, as required. Periodic
orbit theory-based tunneling rate estimates and detailed comparisons
with conventional transition state theory (CTST) at various levels
suggest that the KIE in MA is largely determined by zero-point energy
effects and that tunneling plays a minor role.\\



\section{Atomistic simulations with multipole electrostatics}
    
\section{MTP Force Fields}
\noindent
The combination of computation and experiment is able to provide an
atomistically refined picture of the molecular dynamics of condensed
phase systems on the picosecond timescale of complex solvated
systems. Detailed atomistic simulations can link experimentally
observed dynamics to an underlying structural interpretation of
spectroscopic responses in the frequency- and time-domain of complex
systems in the condensed phase. Experiments are highly sensitive to
the environmental dynamics around suitable probe, and the relevant
dynamics is governed by electrostatic and van der Waals
interactions. In order to provide a realistic description of the
spectroscopy, energetics and dynamics of systems, especially for
solution-phase simulations, the quality of the underlying
computational model needs to be sufficiently accurate.

\noindent
A quantum-mechanical description would present the most rigorous
representation, however the computational limitations restrict the
amount of sampling one would carry out. As an example, for an isolated
chromophores in solution, at least several nanoseconds of MD
simulations are required which corresponds to $10^6$ energy
evaluations to be carried out.This is why one resorts to more
apporximate force-field simulations, which allow extensive sampling of
configurational space. The validity of the underlying computational
model is verified by comparing with reference data from
experiments. Since the relevant dynamics is governed by electrostatic
and van der Waals interactions, multipolar and polarisable force
fields are necessary for the interpretation of time scales and
structural changes at atomistic level. However, the point-charge force
fields are not necessarily inferior compared to MTP parameterization
depending on the molecule considered and the property studied.

\noindent
The utility of atomistic simulations depends on a number of essential
prerequisites. And since the relevant dynamics is governed by
electrostatic and vdW interactions a robust parameterization of the
non-bonded interaction is required. The accuracy of the optimized
parameters is evaluated by comparing the observables obtained from
atomistic simulations to experimental data. A fitting producedure was
recently developed, which facilitates the tedious force fields, and is
made available as a friendly graphics-based interface.


\subsection{Force Field Parameterization}
In conventional force fields, the electrostatic potential is
represented by atom-centered point charges (PC). This approximation
represents the fastest way to describe the charges. Every atom is
simply assigned with one charge located in its core. The value of
these charges are calculated to be as close as possible to the actual
electrostatic potential. However in the case of halogens, the point
charge method presents limitations toward the "sigma-hole"
feature. The sigma-hole is due to the anisotropy of the atom's charge
distribution. It represents a depletion of electronic density
prolonging the C-X bond, concomitant with a build-up on its
sides. This is why in these simulations we will use the multipole
electrostatics MTP, that is a higher-order expansion of the
electrostatic potential, therefore the atoms are described by
monopoles (point charges), dipoles and quadrupoles. Thus, polarization
effects are taken into account by fitting the system's driving forces
on interatomic interaction potentials derived in part from electronic
structure calculations and from experimental data.

\noindent
Nonbonded interactions refer to van der Waals terms and the
electrostatic terms between all atom pairs. The parameterization of
the force-field rely on the optimization of these parameters through
fitting on gas-phase calculations and on experimental data, where the
charges or multipoles are first fit to the {\it ab initio} calculated
electrostatic potential (ESP) and then the nonbonded parameters are
fit to best reproduce thermodynamic properties (density $\rho$, heat
of vaporization $\Delta H_{\rm vap}$ and hydration free energy $\Delta
G_{\rm hyd}$).
 
\noindent
First, PC or MTP parameters of each compound are derived from the
ab-initio electrostatic potential (ESP) of the gas-phase molecule: MP2
level with aug-cc-pVDZ basis set using G09 software. This analysis is
based on the electron density which is used to derive the potential
and thus the multipole expansion. The most widely used method to
obtain atomic multipole moments is the Distributed Multipole Analysis
(DMA).  Second, ESPs are fit to PC / MTP parameters in the first
interaction belt: by choosing fit parameters, running the optimization
and evaluating the error in reproducing the ESP.

\noindent
After the electrostatic optimization, the van der Waals parameters are
scaled to best reproduce the experimental values of density, heat of
formation and hydration free energies. The nonbonded parameters from
the CGenFF\cite{cgenff2012} force field were used as a starting
point. Based on those and multipoles from a GDMA
analysis\cite{Stone05p1128} of the electron density calculated at the
MP2\cite{gordon88mp2}/augcc-pVDZ\cite{dunning89BtoNe,dunning93AltoAr}
level of theory the electrostatic potential (ESP) was fit to best
reproduce the {\it ab initio} density while keeping the point charges
as close to the CGenFF PCs. The MTP model which gives the least value
for root mean square energy (RMSE) between the calculated and fitted
ESP are the ones used for MD simulations.  \\

\noindent
{\it The heat of vaporization} was determined from MD simulations (see
below) according to\cite{junmei2011}
   \begin{equation} 
   \label{heat_eqn}
   \Delta H_{\rm vap}(T) = E_{\rm gas}(T) -  E_{\rm liq}(T) + RT
   \end{equation}
where $E_{\rm gas}$ and $E_{\rm liq}$ are the potential energies of
one molecule in the gas and liquid phases, $T$ is the temperature and
$R$ is the gas constant.  The gas-phase energy is computed from the
minimized energy and the number of atoms, $N$, and constrained degrees
of freedom, $N_{\rm cons}$, in the molecule
   \begin{equation} 
   \label{heat_eqn2}
   \Delta E_{\rm gas}(T) = E_{\rm gas}^{minimized} +
   \frac{1}{2}RT(3N-6-N_{\rm cons})
   \end{equation}

{\it Free energies of hydration} (i.e. solvation free energy in water)
were computed by thermodynamic integration (TI) method. With the use
of thermodynamic cycle, the hydration free energy $\Delta G_{\rm hyd}$
= $\Delta G_{\rm wat}$ - $\Delta G_{\rm vac}$ was calculated, where
$\Delta G_{\rm wat}$ and $\Delta G_{\rm vac}$ corresponds to the free
energy of insertion of the solute in a box of water and vacuum. To
calculate both thermodynamic observables, a separate $NPT$ simulation
was run in which solute molecule was placed in a box of water with
dimensions of 28 \AA\/ each.  $NPT$ simulations were performed using
the Hoover heat-bath method with pressure coupling\cite{feller1995} at
$T = 298$ K, $p = 1$ atm, and the masses of the temperature and
pressure piston to roughly 20\% and 2\% of the systems mass,
respectively. System studied was heated up at constant volume for 40
ps, followed by a 40 ps $NPT$ equilibration using a Langevin damping
coefficient on the piston $\gamma_p=$ 20 ps$^{-1}$, and finally a 40
ps $NPT$ production run.\\
 
\subsection{Applications}
%  { \bf Differences in g(r) and N(r) around the C(CX) atoms between
%    the PC and MTP models provide an explanation for : the difference
%    in the calculated ∆Ghyd between both models Reproducing ∆Ghyd when
%    placing MTP on C(CX)
%   
%   
%   Condensed-Phase Multipole Electrostatics
%   
%   Here we use additive force field: CHARMM
%   with provisions to multipolar electrostatics which has shown to be necessary for the interpretation of time scales and% structural changes at the atomistic level.
%   
%   Introducing the “Fitting Wizard”
%   
%Solvation of Fluoro-acetonitrile (FACN) in water
%
%   Solvent dynamic time scale: Spectral diffusion decay times were
%   obtained from FFCF fits to double or triple exponential decays}

\subsection{conclusion}
Atomistic simulations are suitable to contribute to a detailed
understanding of the molecular dynamics of condensed phase system and
paves the way for structural interpretations of spectroscopic
responses in the frequency- and time-domain of complex systems

\section*{Acknowledgments}
The NCCR MUST (to MM) and the University of Basel are acknowledged.


\bibliography{../literature}

\end{document}
