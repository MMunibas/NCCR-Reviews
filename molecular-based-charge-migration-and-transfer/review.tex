\documentclass[journal=jacsat,manuscript=article]{achemso}

%\usepackage{...}

\bibliographystyle{achemso}

\author{Akshaya K. Das and Markus Meuwly}
\affiliation{Department of Chemistry, University of Basel, Klingelbergstrasse 80, CH-4056 Basel, Switzerland.}
\email{m.meuwly@unibas.ch}

\date{\today}

\title{}

\begin{document}

\begin{abstract}

\end{abstract}

\clearpage

\section{Solvent Dynamics Following Oxidation Reactions in Model Systems}
Solvent molecules around metal complexes are of fundamental and
practical relevance both, in catalysis and in solar energy
conversion.\cite{Balzani96,macchioni,Kalyan98} However, directly
observing and characterizing the influence and role of solvent is
difficult experimentally mainly because of the transient nature of the
process and the finite time resolution of the experiments.

A pivotal system to study solvent reorganization following
photoexcitation (oxidation) is Fe-tris-bipyridine
[Fe(bpy)$_3$]$^{2+}$. Iron-containing complexes constitute an
important and versatile class of metal complexes. Depending on the
strength of the ligands they can exist either in a low (LS) or in a
high spin (HS) state. Specifically, the [Fe(II)(bpy)$_3$] complex
exhibits two spin states: a LS singlet ($^1$A$_1$) and a HS quintet
($^5$T$_2$).\cite{scienceChergui,Gaffney} Between these two states,
excited state charge transfer and spin dynamics takes place (spin
crossover - SCO - dynamics) which can be induced by illumination,
temperature or pressure changes.\cite{Haldrup}

Photoexcitation of the iron-complex leads to a non-equilibrium
preparation of the system from where it relaxes towards a new
stationary state. Such processes can be investigated with state-of-the
art X-ray spectroscopic techniques. However, directly observing and
characterizing the influence and role of solvent is difficult
experimentally mainly because of the transient nature of the process
and the finite time resolution of the experiments.

Atomistic simulations with validated force fields constitute a
complement to such experiments and provide detailed molecular-level
insight into the motions and time scales involved. For the present
situation the VALBOND force field provides the necessary accuracy. It
is based on valence bond theory and capable to more realistically
describe angle bending in metal
complexes.\cite{Landis1,Landis2,Landis3} Unlike the simple harmonic
approximation, VALBOND bending functions capture the energetics at
very large angular distortions and support hypervalent compounds by
means of 3-center-4-electron bonds.\cite{Landis2}

The system considered is one [Fe(bpy)$_3$]$^{2+}$ solvated in explicit
water.\cite{das:2016} Suitable force fields were parametrized and
validated for three states: Fe(II)$\rm _{LS}$, Fe(II)$\rm _{HS}$ and
Fe(III). The actual simulations were started in the Fe(II)$\rm _{LS}$
state and electronic excitation to Fe(III) was induced by
instantaneously changing the force field parameters between the two
oxidation states. Similarly, the relaxation process of Fe(III) to
Fe(II)$\rm_{HS}$ was carried out by changing the force field
parameters from Fe(III) to Fe(II)$\rm _{HS}$. Such perturbations lead
to a non-equilibrium situation from which the system relaxes towards
an equilibrium state.

The analysis of the molecular dynamics trajectories included the
radial distribution functions $g(r)$ between the complex and the
solvent water and the lifetime of the water molecules close to the
complex. It was found\cite{das:2016} that the degree of solvation
decreases on a sub-picosecond time scale and originates from the
excitation to the $^{1,3}$MLCT band. Hence, water expulsion occurs
between [Fe(II)$_{\rm LS}$(bpy)$_3$] and the $^{1,3}$MLCT state. The
process is electronically driven and occurs on a sub-picosecond time
scale. Furthermore it was found that relaxation of the non-equilibrium
ensemble of the $^{1,3}$MLCT state to the equilibrium [Fe(II)$_{\rm
    HS}$(bpy)$_3$] state occurs on the picosecond time scale which
agrees with recent experiments\cite{Majed2015} and the water exchange
dynamics in the inner shells close to the metal center take place on
the picosecond time scale.

    
\section*{Acknowledgments}
The NCCR MUST (to MM) and the University of Basel are acknowledged.


\bibliography{../literature}

\end{document}
