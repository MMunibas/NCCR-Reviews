\documentclass[journal=jacsat,manuscript=article]{achemso}

\usepackage{graphicx}
\usepackage{amsmath}

\bibliographystyle{achemso}


\author{Krystel El Hage} \affiliation[University of Basel] {Department
  of Chemistry, University of Basel, Klingelbergstrasse 80, 4056
  Basel, Switzerland}
\author{Luigi Bonacina, Cederic Schmidt, Geoffrey Gaulier}
\affiliation[University of Geneva] {Department of Chemistry, University of Geneva}
\author{Siri C. van Keulen, Swarnendu Bhattacharyya }
\affiliation[EPFL] {Department of Chemistry, EPFL}
\author{QQQQ}
\affiliation[EPFL] {Department of Chemistry, EPFL}
\author{Majed Chergui}
\affiliation[EPFL] {Department of Chemistry, EPFL}
\author{Peter Hamm}
\affiliation[University of Zurich] {Department of Chemistry, University of Zurich}
\author{Ursula Rothlisberger}
\affiliation[EPFL] {Department of Chemistry, EPFL}
\author{Jean-Pierre Wolf}
\affiliation[University of Geneva] {Department of Chemistry, University of Geneva}
\author{Markus Meuwly} \email{m.meuwly@unibas.ch}
\affiliation[University of Basel] {Department of Chemistry, University
  of Basel,\\ Klingelbergstrasse 80, 4056 Basel, Switzerland}


\date{\today}

\title{Implications of Short Time Scale Dynamics on Long Time
  Processes}

\begin{document}

\begin{abstract}

\end{abstract}

\clearpage

\section{Introduction}
%{\bf Die Idee, die ich dabei verfolge, ist, das Problem aus einer
%  "Energieperspektive" zu diskutieren. Energie laesst sich relativ
%  "einfach" in ein System einspeisen, dann aber nur schwierig
%  kontrollieren, wenn man nicht auch die Energie-Transfer-Kanaele
%  kontrolliert. Dies geschieht in der Regel ueber structural
%  constraints. Diese Denkweise verbindet die Konzepte "Energie",
%  "Struktur" und "Zeit" miteinander.}

Many fundamental processes in chemistry, biology and physiology occur
on time scales slower than microseconds but they have their origin in
dynamics on the the femto- to picosecond time scale. A topical example
is the generic time scale of a chemical reaction in solution which
occurs on a typical time scale of a second. However, the actual
elementary process (bond formation/bond breaking) is a picosecond
process. The ultimate reason for this large span of time scales is the
inefficient coupling of thermal motion of the atoms (primarily
translational degrees of freedom) to the coordinate(s) along which the
reaction progresses.

The purpose of this review is to highlight topical examples in which
information and insight from short-time dynamics has implications for
processes occurring on considerably longer time scales. One such
example is the solvent dynamics around solutes which can be
characterized experimentally at considerable detail on the picosecond
time scale with implications for the solution thermodynamics of the
solute. Other examples include the photophysics and photochemistry of
small peptides or the molecular dynamics of hydrogen-bonded systems.

Triggering events provide energy which can be supplied to a system of
interest in various ways. The most natural origin is thermal energy
which, however, is often inefficient. Other possibilities include
photoexcitation (flash photolysis, electronic excitation), ligand
binding, pH-change, collisional excitation or the excitation of
molecular vibrations. {\bf need suitable citations here.} Following
such a change in the system, processes including catalysis or further
protein-ligand binding can take place. Understanding the link between
cause and effect and following the system's dynamics as a function of
time is of fundamental importance for fully characterizing the
function of a complex system and move towards molecular design.

In situations where multiple elementary steps take place between the
initial preparation and the finally observable state the sequence of
events may also be unknown. Two examples should highlight this
situation.
\begin{itemize}
\item In the homodimeric hemoglobin HbI from {\it Scapharca
  inaequivalvis}
  experiments\cite{Song1993,Royer1990,Knapp2005,Nienhaus2007,Chiancone2000,Eaton1999,choi2010}
  and atomistic simulations\cite{Zhou2003,Knapp2009,MM.scapharca:2016}
  have found that the functionally relevant dynamics between the deoxy
  (T, ligand unbound) and the oxy (R, ligand bound) state is strongly
  influenced by structural changes at the interface and the number of
  water molecules between the two monomers. The most significant
  tertiary structural change concerns the orientation of the
  phenyl-sidechain of Phe97 in which the $\chi$ angle changes from
  $50^\circ$ to $160^\circ$ upon ligand binding. The conformational
  change is accompanied by a change in the degree of hydration of the
  interface in that 17 water molecules are present in the deoxy state
  and only 11 water molecules are found in the ligand-bound
  protein. One open question is whether side-chain rotation drives
  water diffusion or {\it vice versa}.

\item Sensing environmental signals and rapid metabolic response in
  bacteria can be accomplished through so-called "two-component"
  signal transduction systems\cite{Stock2000_b} consisting of a
  histidine protein kinase that transfers a phosphoryl group to a
  conserved aspartate of a response regulator (RR) protein which
  modulates its activity. The diguanylate cyclase PleD of {\it
    Caulobacter crescentus} is such a response
  regulator.\cite{Aldridge1999,Aldridge2003,Paul2004,Hecht1997} The
  protein synthesizes the bacterial second messenger cyclic
  di-guanylic monophosphate (c-di-GMP) \cite{Paul2004}, a molecule of
  great interest which regulates surface-adhesion properties and
  motility in bacteria \cite{Malone2006}. In order to carry out its
  function, several elementary steps (protein dimerization, activation
  through phosphorylation, allosteric (auto)inhibition) need to take
  place.\cite{Jenal2006} However, the sequence in which these
  elementary steps take place is unknown but of fundamental importance
  to understand the mode-of-action of this protein.
\end{itemize}

Reasons for the wide spanning time scales between cause and effect are
inefficient coupling to transfer energy between different degrees of
freedom, the (slow) conformational dynamics which is a ``search
problem'' on a high-dimensional, rough potential energy surface or
energy dissipation due to friction. All these processes lead to a
marked slowdown in transmitting the primary information into a
productive channel. For processes in solution a further complication
are entropic effects which can lead to additional slowdown. Finally,
it is also possible that the ``initial'' and ``final'' state are
connected through multiple pathways as in the photocycle or
bacteriorhodopsin\cite{korenstein:1978,hendriks:1999} or ligand
rebinding in neuroglobin.\cite{MM.cei:2013}

\begin{figure}[t]
\centering
\includegraphics[width=0.9\textwidth]{fig/model.png}
\caption{Illustration of a photoinduced process (red arrow) and the
  ensuing dynamics along a conformational dynamics on the excited
  state potential energy surface spanning several orders of
  magnitude. The orange curve tracks the excited state dynamics in
  different basins and on different time scales.}
\label{fig:model}
\end{figure}




\section{Systems Dynamics from Femtoseconds to Milliseconds}


\subsection{Small Gas Phase Systems: Sulfuric Acid (H$_2$SO$_4$)}
For reactions in the atmosphere unimolecular dissociation often plays
a key role. Those kind of processes can be initiated by overtone
excitation, for example the photofragmentation of sulphuric acid
(H$_2$SO$_4$) in the strato- and troposphere which can be induced by
OH-stretching excitation. The dissociation products (water and
sulphur-trioxide) are of importance to the tropospherical aerosol
layer formation, which plays an important role for the climate and
weather\cite{vaida.sci.2003.vibphotodis}.\\

\noindent 
Theoretical investigation of the reaction by Yosa \textit{et
  al.}\cite{reyes.pccp.2014.msarmd} were performed utilising
MS-ARMD\cite{nagy.jctc.2014.msarmd}. This efficient surface crossing
algorithm that combines (fitted) empirical force fields by a
time-independent switching function to form a global potential energy
surface (PES) is capable of treating several reaction pathways. This
is necessary in the investigation of H$_2$SO$_4$, since the
fragmentation pathway is in competition with intramolecular H-transfer
reaction of similar barrier height. From several thousand independent
($NVE$) trajectories with a maximum simulation time of 1 ns it was
found that the fast OH overtone vibration (period $T \approx 1-3$ fs)
induces elimination on the pico- to nanosecond time scale, depending
on the excitation state of the OH-stretching. Final state analysis of
the resulting trajectories results in an energy distribution that can
aid the identification of the products.\\

\noindent
Since no experimental verification for the OH-stretching overtone
induced photofragmentation of H$_2$SO$_4$ exists, based on the low
vapour pressure of the molecule and its equilibrium with its products
in gas phase, derivatives of sulphuric acid have been considered for
providing experimental proof of the reaction mechanism. Therefore
MS-ARMD simulations on sulphurchloridic acid (HSO$_3$Cl) were
performed\cite{reyesbrickel.pccp.2016.msarmd}. The calculation were
performed on a global PES, consisting of three, chemically equivalent,
states describing the possible conformations of HSO$_3$Cl and one
state describing the elimination product, SO$_3$ + HCl. The PES was
parametrised to several thousand MP2/6-311G++(2,2p) reference
points. 5000 independent trajectories with a simulation time of 2.5 ns
give rise to up to 98\% of elimination. For low excitation energies
($\nu_{\text{OH}}$ = 3 and 4) the yield of photofragmentation is lower
than the percentage of trajectories that show only H-transfer (thereby
leading to complete intramolecular vibrational energy redistribution
(IVR)). The direct elimination path for $\nu_{\text{OH}}$ = 3 yields
only 0.2\% in comparison to 10\% IVR, for example. High excitation
energies ($\nu_{\text{OH}}$ = 5 and 6) have an increased yield of
elimination. The photodissociation occurs here mostly by indirect
elimination, i.e. elimination preceded by H-transfer. These two
different ways of elimination (direct and indirect) occur on different
time scales (see figure \ref{fig:vibphotodis}) and lead to a
difference in the vibrational energy distribution of HCl. It was
predicted that for direct elimination the diatomic fragment peaks
around $j = $ 10, while for indirect elimination it peaks at slightly
higher values. Overall it can be seen that the reaction occurs on the
sub-ns to ns time scale, which is faster than collisionally induced
quenching. The process of OH-stretching overtone induced
photodissociation should therefore play a role in atmospheric
chemistry and a deeper understanding can aid in, e.g. calculations of
the climate.




\begin{figure}
\includegraphics[width=\textwidth]{fig/reaction-path_time.png}
\caption{Schematic representation of the possible reaction paths of
  HSO$_3$Cl, including approximate time scales for an high excitation
  energy ($\nu_{\text{OH}} = 5$ and 6).}
\label{fig:vibphotodis}
\end{figure} 
    

\subsection{Medium Sized Gas Phase System: Dynamics in HG$_3$W Observed over 8 Orders of Magnitude in Time}
L. MacAleese and co-authors investigated the process of Proton Coupled
Electron Transfer (PCET) in a silver-containing metal-peptide cation
synthetized from histidine (H), glycine (G) and tryptophan (W):
[HG3W+Ag]+. Very notably, the authors were able to observe an overall
PCET dynamics spanning 8 orders of magnitude in time.  The experiment
is based on the combination of mass spectroscopy and pump-probe
optical spectroscopy. After being electrosprayed, the sample is
trapped for 200 ms in a high-pressure (5 mTorr) chamber inside the
mass spectrometer. The device is custom modified and features an
optical aperture to let the laser beams in.  To cover the extremely
long time span of the process investigated, both femtosecond lasers
with optical delay lines and nanosecond lasers with electronic
time-synchronization are employed.



\begin{figure}[h]
\centering
\includegraphics[width=0.9\textwidth]{fig/wolfjacsf1.png}
\caption{Schematic representation of PCET dynamics in [HG$_3$W +
    Ag]$^+$ metal−peptide complexes. Irradiation at 266 nm initiates
  an electron transfer from tryptophan to Ag$^+$ , leading to the loss
  of Ag.  The electron transfer is followed by a proton transfer from
  tryptophan to histidine with formation of a distonic ion. The
  peptide structures are schemes and do not correspond to calculated
  structures.}
\label{fig:fig1jacs}
\end{figure}


By analyzing the time-resolved traces of i) silver-containing against
silver-free ions over the first 30 ps and ii) the evolution of the
fragmentation yield of the radical peptide [HG3W]$\cdot ^+$ up to 1
ms, the authors could monitor electron and proton transfer on their
respective time scales. The final picture which emerges from this work
is the following: a $\pi \rightarrow \pi*$ transition of tryptophan is
firstly excited upon 266 nm photo-excitation, this in turns leads the
system onto an electron transfer state where Ag is neutral and both
charge and radical are localized on tryptophan
(Fig. \ref{fig:fig1jacs}). After this short dynamics (3.5 ps), a
proton is transferred from tryptophan to histidine with a time-scale
of the order of hundreds of microseconds. Molecular dynamics
simulations indicate that, although the rate of formation of a
proton-transfer reactive structure can be as short as a few
nanoseconds for such a structure, this time-scale is strongly
influenced by the initial peptide conformation. In particular, the
presence of Ag$^+$ in [HG3W+Ag]$^+$ entails a series of structural
changes including un unfavorable orientation of the tryptophan side
chain with respect to histidine and the presence of several H-bonds,
which lock the initial conformation hindering the formation of a
proton-transfer structure (Fig. \ref{fig:fig5jacs}). Indeed, the
calculations based on this conformation do not predict the occurrence
of proton-transfer up to 8 $\mu$s, consistently with a major
slowing-down of the proton-transfer process as observed
experimentally.



\begin{figure}[h]
\centering
\includegraphics[width=0.9\textwidth]{fig/wolfjacsf5.png}
\caption{IndoleNH−imidazoleN distance distribution in HG$_3$W$^{.+}$
  over 50 independent runs of 10 ns each, starting from an extended
  structure (A) and two compact structures (B and C) obtained from the
  [HG$_3$WAg]$^+$ complex. (D) Same but over 8 independent runs of 1
  μs each, starting from structures in C. Insets show superpositions
  of structures from MD trajectories.}
\label{fig:fig5jacs}
\end{figure}





\subsection{Small Solvated Systems: PDZ Domain and Insulin}

\subsubsection{Local Perturbations to Mimic Allosteric Dynamics}
Allostery is the coupling of conformational changes between two
separated sites of a protein. Allostery is an important mechanism that
Nature uses regulate the affinity of certain substrates to a protein,
thereby controlling metabolism. According to the conventional view of
allostery, a conformational change of the protein (that might however
be very small~\cite{Nussinov15}) is the source of a signal, but it
should be noted that other mechanisms have been proposed as well that
work exclusively with dynamical properties~\cite{cooper84}. In any
case, binding of a ligand at a so-called \textit{allosteric site}
increases (or decreases) the affinity for a substrate at a distant
\textit{active site}. Hence, an allosteric protein can be viewed as a
``transistor'', and complicated feedback networks of many such
switches ultimately make up a living cell~\cite{alon07}.

The ``textbook'' explanation of allostery is depicted in
Fig.~\ref{figAllostery}a. From the point of view of allosteric
regulation, this picture might appear sufficient, i.e., what we need
to know is the change of binding affinity depending on whether or not
a ligand is bound to the allosteric site. This is essentially the
level of description of the MWC model to explain the cooperative
binding of oxygen to hemoglobin - the prototype example of allosteric
regulation~\cite{Eaton99}. However, from a microscopic, atomistic
point of view, Fig.~\ref{figAllostery}a does not say much about how,
why and how quickly the transition occurs. Given the complexity of the
regulation network in a living cell, the switching speed, albeit
probably being fast on the biological timescale, might not be
irrelevant.


The equivalent of the level of description of Fig.~\ref{figAllostery}a
applied to the protein folding problem would reduce the latter to a
folding free energy. Again, from a higher biological perspective, this
is often sufficient, as all we need to know is how stable a protein is
and what its folded structure is. Nevertheless, the protein folding
problem has, of course, been tackled on a much more microscopic level,
leading to concepts such as the folding funnel, rugged energy
landscapes, down-hill folders, folding networks, etc.~\cite{brooks98,
  dill08,kubelka04,Gruebele99,krivov04,caflisch06}. The emerging view
of allostery works on the hypothesis that any protein exists as a
conformational ensemble, and that a conformational change upon ligand
binding is the result of a shift in populations within that
ensemble~\cite{gunasekaran04, tsai14}.

\begin{figure}[h]
\centering
\includegraphics[width=0.6\textwidth]{fig/figAllostery.pdf}
\caption{(a) The ``textbook'' explanation of allosteric regulation: A
  ligand (orange) binds to the allosteric site of a protein, thereby
  changing the binding affinity for a substrate at a distant active
  site (red). (b) A PDZ2 domain with an azobenzene-moiety linked
  across the ligand binding groove (PDB entries 2M0Z and 2M10)}
\label{figAllostery}
\end{figure}

With the one exception of hemoglobin~\cite{Eaton99}, the dynamics of
the structural transition giving rise to allosteric regulation has not
been investigated with high time resolution. Despite the fact that it
is not its natural function, hemoglobin is photo-switchable in its
natural form through the photo-dissociation of a heme-CO, enabling
time-resolved experiments with essentially unlimited time
resolution. It has been found that tertiary conformational changes
occur in the time range from 1~ns to 1~$\mu$s in a highly
non-exponential manner, whereas quaternary changes are slower.

We recently set out to develop tools to initiate an allosteric
response in a protein that is not \textit{per se}
photoswitchable~\cite{buchli13,waldauer14,buchenberg14}, applying a
concept that has first been introduced by Woolley and
coworkers~\cite{kum00, woolley05, beharry11}. To that end, an
azobenzene-derivative is used as a photoswitch, which can be
isomerized between its \textit{cis} and a \textit{trans} conformer
with light of different wavelength on a very fast 1~ps timescale. The
switch is chemically designed such that it can be covalently linked to
virtually any position at the protein surface via two
cysteines. Hence, upon photo-isomerization, one may apply a force
between two points of a protein in a extremely well controlled
manner. In the past, the concept has mostly been used to initiate
folding/unfolding of small peptides~\cite{ihalainen08,schrader07} or
proteins~\cite{zhang09}, and we now set out to apply it to a bi-stable
system, i.e., switching between the two states of an allosteric
protein.

We chose the second PDZ (PDZ2) domain from human tyrosine-phosphatase
1E (hPTP1E) for these studies, which has been demonstrated to possess
allosteric properties~\cite{fuentes04}, and which has served as a
model system for allostery for a long time. Possible signal
transduction pathways and mechanisms have been widely investigated by
md simulations as well as bioinformatics approaches~\cite{Ota05,
  DeLosRios05,sharp06, Dhulesia08, Kong09, Gerek11,
  Cilia12,lockless99,suel03,Chi08}. The PDZ2 domain is a small 96
residue protein with a binding groove between the $\alpha_2$-helix and
the $\beta_2$-strand (see Fig. \ref{figAllostery}b). Binding of a
ligand results in a small but measurable structural change with an
rmsd of the binding groove of $\approx$0.4~\AA, according to X-ray
crystallography~\cite{zhang10}.


\begin{figure}[h]
\centering
\includegraphics[width=0.6\textwidth]{fig/figResponse.pdf}
\caption{Response of the PDZ2 domain after perturbing the binding
  groove with the help of a azobenzene-crosslinker, measured by
  transient IR spectroscopy. The red data show the response of the
  protein as a whole via the amide I band, and the green data that of
  a vibrational mode localized on the azobenzene-crosslinker. Adapted
  from Ref.~\cite{buchli13} with permission.}
\label{figResponse}
\end{figure}

Transient IR spectroscopy has been used to study the response of the
protein upon photo-triggering the azobenzene-crosslinker
(Fig.~\ref{figResponse})~\cite{buchli13}. The initial event, i.e., the
actual photo-isomerization the azobenzene-moiety, is a barrierless
photochemical reaction that occurs on a few 100~fs
timescale~\cite{naeg97}. It triggers a cascade of events in the
protein, which cover even orders of magnitudes in time. Three phases
can be distinguished: During phase I up to $\approx$40~ps, the heat
deposited into the protein as a result of the photo-isomerization of
the cross-linker, which dissipates about 3 eV of vibrational energy,
cools into the solvent. Next, during phase II from $\approx$40~ps to
$\approx$100~ns, the binding groove opens, observed via the response
of a vibrational mode localized on the azobenzene-crosslinker
(Fig.~\ref{figResponse}, green). Finally, during phase III beyond
$\approx$100~ns, more remote parts of the protein adapt to the
perturbation, seen as a delayed response of the amide I band which
reports on the structure of the protein backbone
(Fig.~\ref{figResponse}, red).

\begin{figure}[h]
\centering
\includegraphics[width=0.6\textwidth]{fig/figMechanism.pdf}
\caption{Mechanism of the allosteric response.}
\label{figMech}
\end{figure}

In Refs.~\cite{buchli13,waldauer14}, we have mostly focused on phase
II, i.e., the opening of the binding groove, which occurs in a highly
non-exponential manner and by itself covers 3.5 orders of magnitudes
in time. Non-exponential protein dynamics have been discussed
extensively, for instance in the context of ligand (CO) dissociation
and rebinding in hemoglobin or myoglobin~\cite{Frauenfelder91,
  frauenfelder09}. Two limiting scenarios are typically discussed: A
parallel process is characterized by a distribution of exponential
decay processes, originating from a distribution of barrier heights in
an inhomogeneous ensemble of proteins (Fig.~\ref{figMech}a). In that
case, individual single-molecule trajectories would still behave as
two-state system with either a closed or an open binding groove, and
one would observe essentially sudden jumps between these two states
(Fig.~\ref{figMech}c, black). The distribution of jump times would be
non-exponential, revealing a non-exponential response after ensemble
averaging (Fig.~\ref{figMech}c, red). In the opposite limit, the
system diffuses on a rugged, highly dimensional energy landscape
(Fig.~\ref{figMech}b), which commonly leads to non-exponential
response as well~\cite{jaeckle86}. In this case, single-molecule
trajectories would essentially be equivalent with the average, apart
from statistical noise, without big jump (Fig.~\ref{figMech}c, black
versus red). This is indeed what has been observed in accompanying MD
simulations~\cite{buchli13}. We furthermore found that water friction
is in part the source of the ruggedness of the protein energy
landscape~\cite{buchli13,waldauer14}. That result agrees with the view
that the allosteric response is related to a shift in populations
between the substates of an ensemble of protein
conformations~\cite{gunasekaran04, tsai14}. The response thus shares
many properties with downhill folding~\cite{kubelka04,yang03,sadqi06}.

Fig.~\ref{figResponse} also illustrates a limitation of our work so
far. While by chance we could isolate one vibrational mode localized
on the photo-switch (Fig.~\ref{figResponse}, green), which enabled us
to draw very specific conclusions on the binding groove
dynamics~\cite{buchli13,waldauer14}, the response of the amide~I band
(Fig.~\ref{figResponse}, red) averages over the whole protein without
any site-specific information. While MD
simulations~\cite{buchenberg14} suggest that these slower processes
are related to conformational changes in the floppy parts of the
protein (i.e., termini and loop regions), and as such might actually
be related to the allosteric signalling in ``remote^^ parts of the
protein, we do not yet have any direct experimental support for
that. We currently work on the incorporation of non-natural
amino-acids containing isolated vibrational labels, similar to
Ref.~\cite{bloem12}, which eventually should allow us to retrieve
site-selective information from transient IR spectroscopy in a
versatile manner.




\subsubsection{Halogenated Insulin: Functional Studies Based on Accurate Models from Short Time Simulations}
Linking time scales in a different context is afforded by a discussion
of recent work related to the dynamics and energetics of halogenated
insulins. Halogenation has been found to modify the aggregation and
binding behaviour of insulin under physiological conditions. In order
to lay out a roadmap for further modification and optimization of the
hormone a molecular-level understanding of the energetics and dynamics
of insulin in various aggregation states is required. The system of
interest is an iodinated insulin in which a hydrogen atom on the ring
of TyrB26 was replaced by an iodine atom. Mixed quantum
mechanical/molecular mechanics simulations would, in principle, be the
method of choice for such a problem. However, given the large number
of electronis, the high level of electronic structure calculations
required and the extended time scales on which the system needs to be
followed for meaningful comparison with experiment, such an approach
is currently not viable. Instead, an improved description of the
intermolecular interactions based on multipolar force fields was
envisaged and using of this parametrization to follow the dynamics and
energetics of insulin in different aggregation states and in contact
with the microreceptor was envisaged.\\

\begin{figure}[h]
\centering
\includegraphics[width=0.6\textwidth]{fig/param-scheme.pdf}
\caption{Flowchart illustrating the FW. The left panel corresponds to
  fitting the MTPs to the ESP as obtained from the electronic
  structure calculations. The right panel summarizes refinement of LJ
  parameters for optimal reproduction of selected thermodynamic
  observables.}
\label{fig:param}
\end{figure}

\noindent
For this, model compounds were first parametrized from {\it ab initio}
calculations and fitted to available experimental data which included
hydration free energies $\Delta G_{\rm hyd}$, heat of vaporization and
the pure solvent density. The electrostatic multipole model
(MTP)\cite{bereau2013} was obtained for the phenolic ring of TyrB26
and iodophenolic ring of I-TyrB26. Atomic multipoles are assigned to
all heavy atoms (but not the hydrogens). The parametrization protocol
followed a recently developed strategy which includes optimization of
multipole moments to best represent the electrostatic potential and
van der Waals parameters to correctly describe experimental solution
phase data, including the hydration free energy of
iodophenol.\cite{toolkit:2016,toolkit:2017} MTPs were derived from ab
initio calculations at the MP2/aug-cc-pVDZ level of theory using
GAUSSIAN09. The iodine atom was treated by the aug-cc-pVDZ-PP basis
set with an effective core potential. A parametrization based on such
data can be expected to describe the most relevant interaction modes
between solute and solvent in a meaningful way to allow quantitative
simulations for the situation in the protein. Accurate hydration free
energies are sensitive to the pico- and nanosecond dynamics around the
solute.

\begin{figure}[h]
\centering
\includegraphics[width=0.6\textwidth]{fig/esp.png}
\caption{Top: Electrostatic-potential (ESP) surface maps of phenol,
  iodophenol and iodophenyl at the 0.001e bohr-3 isodensity. The color
  scale of the ESP ranges from $-2.12(-2)$ (red) through 0 (green) to
  $2.12 e(-2)$ (blue). In the upper row the iodine (facing the viewer)
  exhibits the effect of the electron-donating –OH on the
  $\sigma$-hole. The lower row shows effects of iodine on the
  $\pi-$system of the phenol ring.  The angle $\beta$ represents the
  $\sigma-$hole size as delimited by black dashed lines. $\delta+$ and
  $\delta-$ represent respective regions of positive and negative
  charge around the iodine. Bottom: ESP contours of iodophenyl (left)
  and 2-iodophenol (right), at different isovalues, calculated in the
  plane of the aromatic ring. The black dashed arrow indicates
  directionality of the C-I bond. }
\label{fig:esp}
\end{figure}


\noindent
In order to guide the experiments, atomistic simulations of insulin
dimer and insulin monomer complexed to the $\mu$IR were carried
out. These studies indicated that replacement of one hydrogen atom by
an iodine at the ortho position of TyrB26 leads to structural
rearrangements at the dimerization interface and enhanced binding to
the receptor due to several favourable interactions. These findings
were subsequently confirmed by X-ray crystallography and affinity
measurements.\cite{insulin:2016} In particular, the simulations
predicted insertion of the large iodine atom (atomic radius ca
two-fold larger than that of carbon) within an internal cleft between
the A- and B chains (see Figure \ref{fig:insulin}). Such accommodation
requires a specific reorientation of the B26 side chain. In these
simulations the internal location of the iodine atom within a protomer
was found to be compatible with native-like assembly of the dimer
interface within the hexamer with subtle reorganization of successive
aromatic-aromatic interactions. The following predictions were then
verified by determining the crystal structure of a 3-I-TyrB26-insulin
hexamer at 2.3 \AA\/ resolution. The nonpolar packing of the B26
iodo-aromatic ring is thus reminiscent of the packing in the WT system.

\noindent
The binding mode of 3-I-TyrB26 towards the $\mu$IR was also probed
based on the structure of the monomer-$\mu$IR complex. Remarkably,
these simulations predicted formation of favorable halogen bonding and
halogen-directed hydrogen bonding between the modified B26 ring and
the receptor. At this point it is important to mention that
accompanying simulations with point charge models only find one
favourable contact between iodine and the environment at the receptor
interface which highlights the need for improved electrostatic models
when considering halogenated compounds. Such directional electrostatic
interactions, exploiting respectively the $\sigma$-hole of the halogen
and its electronegative equatorial band, have previously beed observed
in crystal structures of specific complexes between proteins and
iodinated ligands.

\begin{figure}[h]
\centering
\includegraphics[width=0.9\textwidth]{fig/insulin-modes.png}
\caption{Structure of Insulin and receptor domain. Left panels: The
  monomeric hormone (A and B chains, top panel) forms zinc-free dimers
  via anti-parallel association of B chain $\alpha$-helices and
  C-terminal $\beta$-strands (brown, middle panel); two zinc ions then
  mediate assembly of three dimers to form classical hexamer (bottom
  panel). Right panels: $\mu$IR domain accompanied with an overlay
  illustrating insulin in its classical free conformation bound to one
  of the site to the microreceptor.}
\label{fig:insulin}
\end{figure}



\subsection{Solvated Medium Sized Systems: Ligand Binding in Myoglobin}
Our group has been focusing on new methods to probe the electronic and
structural dynamics of various biological and chemical samples. In
particular, the implementation of the first ultrafast 2D set up in the
deep-UV has been a game changer.\cite{aubock:2012,aubock2:2012}

As far as proteins are concerned, this set-up allowed us to unravel
hitherto unknown electron-transfer (ET) processes from photoexcited
tryptophan to the heme in ferric myoglobins (Mb),\cite{consani:2013}
and in deoxy-Mb\cite{monni:2015} while fluorescence resonant energy
transfer (FRET) has been assumed so far. The time scales for ET is
about 20-30 ps, which is relatively slow, but the mechanism is still
unclear. We recently extended the above studies to ligated Mbs, but in
order to avoid the interference of time scales due to ET and to ligand
dynamics we used an IR probe. The results vary remarkably with the
ligand: with CN a reduction of the Fe atoms is observed but some
degree of electron density is on the porphyrin, with NO the electron
goes to the latter, while with CO, the electron is predominantly on
the porphyrin but further CO ligand dissociation from the porphyrin
anion occurs on a time scale of approx. 200 ps, typical of protein
fluctuations.\cite{monni:prep}

Such time scale have been inferred from studies of ligand
recombination as those performed using visible or IR probe, but no
direct evidence that they are caused by protein motions was
available. In a recent ps X-ray absorption spectroscopic study, we
showed that indeed, such long time scales are due to the return of NO
ligand to the Fe atom of the porphyrin,\cite{silatani:2015} which is
facilitated by protein fluctuations.

As far as molecules are concerned, we had already shown that the
intersystem crossing (ISC) rate in diplatinum complexes can be tuned
by solvent effects.\cite{chergui.jacs.2011} However, the fastest time
was in the 10 ps range. Recently, we unraveled a dramatic acceleration
of the ISC in some solvents, to the point that a transfer of
vibrational coherence from the singlet to the triplet state was
observed.\cite{monni:2017,monni:rev} This acceleration was
rationalized by quantum mechanical MD simulations showing that
strongly solvated and mixed high lying excited state serve as the
doorway for the ISC, which is in the order of 700 fs. Over the past
few years, we have shown that ISC times in metal complexes span values
ranging from a few tens of fs to several tens of ps, but that the
so-called “heavy atom effect” is not verified. While the spin-orbit
coupling constant of the metal atom is crucial, other considerations
such as density of states and the structural dynamics play a very
important role, too.\cite{chergui:2012,chergui:2015}


\section{Large Solvated Systems: Rhodopsin}

\subsection{Experimental Studies - Do you want a specific title, JP?}
The vision process, which entails a long cascade of events going from
initial photon absorption to nerve-impulse generation, is triggered by
the rhodopsin-bound 11-cis-retinal to all-trans retinal isomerization
(see Fig. \ref{fig:wolf1}). The ultrafast investigation of this
primary vision step started with the pioneering paper by the Shank
group where the arrival in the isomerization state in less than 100 fs
was time-resolved using a transient absorption
scheme.\cite{schonlein:1991} Shortly after, Wang et al. showed
wavepacket oscillations in this molecule demonstrating that coherence
is preserved for at least 2 ps after photo-excitation despite the
passage through the conical intersection leading to
isomerization.\cite{wang:1994} Gerber’s and Cerullo’s groups
demonstrated that it was possible to transfer the excited population
back to the 11-cis ground state by stimulated emission before the
transition through the conical transition takes
place.\cite{polli:2010,gerber:2006}\\



\begin{figure}[h]
\resizebox{!}{0.6\textwidth}{\includegraphics[width=\textwidth]{fig/wolf1.png}}
\caption{}
\label{fig:wolf1}
\end{figure}


In 2005, Miller and Prokhorenko provided an experimental demonstration
of optimal quantum control of the retinal photo-isomerization in
bacteriorhodopsin.\cite{miller:2005} The authors showed that by
applying an optimally shaped laser pulse retrieved using a
closed-feedback approach, it was possible to increase the retinal
photo-isomerization yield by 20\% as compared to a Fourier limited
femtosecond pulse. Conversely, an anti-optimal pulse shape reduced the
photo-isomerization efficiency by 20\%. Miller’s results demonstrated
that the primary vision step could be modulated by the spectral phase
of light. However, these works were all performed on molecules in
solution, while the possibility of extending the phase-sensitive
interaction with the primary vision step to the overall vision process
in living beings was far from being assessed.


\begin{figure}[h]
\includegraphics[width=\textwidth]{fig/wolf2.png}
\caption{        }
\label{fig:wolf2}
\end{figure} 


Moving on this pathway, Schmidt et al. have recently investigated
whether it is possible to manipulate the photo-isomerization yield of
photoreceptor molecules in a live mouse by modulating the spectral
phase of a green femtosecond light pulse and record the electric
signal generated by the retina. Their experiment is based on a KHz
amplified Ti:Sapphire laser system coupled with a noncollinear
parametric amplifier generating 50 fs pulses at 535 nm. A pulse shaper
device is used to act upon the spectral phase function generating
positively and negatively chirped pulses ranging from -1300 to 1300
fs. The beam is sent onto the eye of the anesthetized mouse by passing
through an index-matching medium to correct for the eye curvature. The
experimental read-out relies on electroretinography (ERG). Namely, the
visual response of an anesthetized animal is cacquired by three
electrodes placed respectively in contact with the irradiated eye, the
forehead, and under skin in the tail (for ground voltage
reference). The signals are first frequency filtered and
pre-amplified. Successively, after being processed by a differential
amplifier, the resulting signal is fed into a fast oscilloscope and
converted into a digital trace. So far, the researchers limited their
investigation to the effect of positively and negatively chirped
pulses. The results are presented in Fig. \ref{fig:wolf2}. In these
ERG traces, one can recognize the characteristic a- (negative at early
times) and b- (positive, at later times) waves. The former is mainly
attributed to cone activity and the latter to second-order retinal
neurons from cones. The results indicate that there exist significant
differences among the traces obtained with different pulses, pointing
unambiguously to the fact that a modulation of photo-isomerization can
induce effect at the end of the cascade of events leading to the
retina signal generation.


\subsection{Rhodopsin: femtosecond \emph{cis-trans} isomerisation induces
  structural rearrangements on the millisecond timescale}

In the outer segment of rod cells in the retina of vertebrate eyes,
the protein rhodopsin is present in high concentrations
\cite{liang2003organization}. In dim light, the membrane-embedded
rhodopsin is able to convert photons into a signal for the brain,
which results in eyesight. Rhodopsin belongs to class A of the
G-protein-coupled receptor (GPCR) family, which has a large impact on
the function of the human body as these membrane proteins enable
communication between the intracellular and extracellular side of a
cell \cite{lagerstrom2008structural}. Class A G-protein-coupled
receptors consist of seven trans-membrane helices, three intracellular
and three extracellular loops as well as a C-terminus and an
N-terminal region (Fig. \ref{intro_rhodopsin2}d). Active GPCRs are
able to stimulate or inhibit several proteins in the cytosol via
interactions with G proteins in the intracellular region
\cite{ritter2009fine}. For several decades rhodopsin has been
intensely studied as a prototype system to understand the activation
process of GPCRs. Besides being viewed as an example protein for other
class A GPCRs, rhodopsin bears also other properties of interest, such
as low basal activity, a covalently bound chromophore in the active
site and a high yield of around 65\% for the photoconversion of the
chromophore's inactive 11-\emph{cis}-retinal state to its active
\emph{trans} configuration (Fig. \ref{intro_rhodopsin2}a)
\cite{kim2001wavelength}.



The \emph{cis-trans} isomerisation of the chromophore initiated by
photon exposure is the trigger for rhodopsin's conversion from its
inactive state to its active metarhodopsin II conformation that
interacts with G proteins in the cytosol. The ultrafast
\emph{cis-trans} photoisomerisation takes place on the femtosecond
timescale in the extracellular part of the protein, while the
structural rearrangements of rhodopsin's intracellular region that
lead to the active state, occur on the millisecond timescale
(Fig. \ref{intro_rhodopsin2})
\cite{pan2001chromophore,jager1997chromophore,jager1997time,ganter1988rhodopsin,struts2011retinal}. Hence,
eight orders of magnitude in time scales are involved to activate the
entire protein due to the transduction of the external signal from the
extracellular to the intracellular region. X-ray crystallography
experiments have elucidated the structure of inactive (dark state)
rhodopsin as well as several intermediates that lead to the fully
active structure
\cite{li2004structure,choe2011crystal,okada2004retinal,salom2006crystal,deupi2012stabilized,xraybatho}. However,
not all intermediates of the activation pathway have been resolved and
protonation reactions that take place during rhodopsin activation are
difficult to understand via experimental data alone. Therefore,
besides experimental studies, a significant number of computational
studies have investigated the chromophore's configurational changes
during protein activation as well as the impact of the structural
change of the chromophore on the active site and the rest of the
protein to obtain a better understanding of the chromophore's
conversion from an inverse agonist to an agonist
\cite{altun2008mechanism,mertz2011steric,saam2002molecular,hornak2010light,neri2010role,rohrig2005solvent}.

Early studies of the photoisomerization reaction dynamics based on a
restricted-open shell Kohn-Sham approach \cite{rohrig2004molecular}
showed that the \emph{cis-trans} isomerisation of the 11-\emph{cis}
retinal moiety that initiates rhodopsin activation is highly impacted
by the shape of the non-polar active site. While in solution,
\emph{cis-trans} isomerisation can also occur around other bonds than
the C11-C12 bond
(Fig. \ref{intro_rhodopsin2})\cite{logunov1996excited,tsukida1977simultaneous},
rhodopsin's tight active site (Fig. \ref{intro_rhodopsin2}b, c)
induces a pre-twist of the C10-C11-C12-C13 dihedral angle, which
favours highly selective isomerisation around
C11-C12\cite{rohrig2004molecular,rohrig2002early}. Within few hundreds
of femtoseconds that system relaxes from the first excited state via a
conical intersection to the ground state. This process can be
infuenced and tuned by an external field (XXXHERE WE WILl ADD SOME OF
THE RESULTS ON MCTDH QUANTUM DYNAMICS SIMULATIONS WITH LOCAL AND WITH
OPTIMAL CONTROLXXXX. The subsequent structural relaxation can be
followed via classical molecular dynamics simulations with
force-matched force fields\cite{doemer2013generalized} up to the
microsecond time
scale\cite{rohrig2002early,neri2010role,vankeulenrhodopsin} and
distinct intermediates can be detected. They show that after
photoisomerisation, a strained \emph{all}-trans configuration of the
chromophore, photorhodopsin, gradually alters its structure and
orientation in the active site to a more relaxed conformation. The
first short-lived intermediate after a few picoseconds is
bathorhodopsin in which the chromophore has released strain, but still
includes a highly distorted \emph{all}-trans
configuration. Bathorhodopsin can form an equilibrium with a
blue-shifted intermediate (BSI) on the nanosecond time scale, which
then leads to a more relaxed conformation called lumirhodopsin after a
few hundred nanoseconds. The alterations in the chromophore's
configuration and orientation during relaxation induce spectral shifts
between the several short-lived intermediates that are sampled after
\emph{cis-trans} isomerisation \cite{sandberg2014low}. These computed
vertical excitation spectra are in excellent agreement with
experimental results. The molecular factors that are responsible for
this spectral tuning have been of special interest and many different
possibilities have been proposed during the years. By applying machine
learning algorithms to a large comprehensive data set generated via
classical and mixed quantum mechanical/molecular mechanical (QM/MM)
simulations , it was recently possible to identify ta minimal set
molecular descriptors in an unbiased way. This study shows that the
principal factors responsible for the spectral tuning are
intramolecular structural features such as the bond length alternation
but also few descriptors that capture the relative orientation of the
chromophore with respect to the active site pocket
(Fig. \ref{intro_rhodopsin2}a)
\cite{campomanes2014origin,valsson2013rhodopsin, vankeulenrhodopsin}.

\begin{figure}[h]
\begin{center}
\resizebox{!}{0.6\textwidth}{\includegraphics{fig/intro_rhodopsin_extended_2_3.png}}
\caption{Rhodopsin's activation mechanism. (a) 11-\emph{cis} to
  all-\emph{trans} photoisomerisation and later deprotonation of
  covalently bound retinal in rhodopsin due to exposure to light. (b)
  View from the extracellular region of the retinal moiety in the
  active site. Courtesy of \cite{vankeulenrhodopsin}. (c) View of the
  retinal moiety, turned 90$^{\circ}$ compared to image (b), with the
  Lys296 part located in front. Courtesy of
  \cite{vankeulenrhodopsin}. (d) X-ray structure of inactive rhodopsin
  (PDB code 1U19). The black lines indicate the location of the cell
  membrane. The yellow circle show the location of the G-protein
  interaction site. (e) Alignment of the active and inactive
  conformation of rhodopsin, viewed from the intracellular
  side. Corresponding PDB codes are listed in the same colour as the
  colour of the protein structure. The main structural changes due to
  activation are indicated by yellow arrows. (f) X-ray structure of
  inactive rhodopsin (PDB code 1U19) in which specific regions that
  are especially affected by retinal's \emph{cis}-\emph{trans}
  isomerisation are highlighted. The retinal moiety is shown in
  orange.}\label{intro_rhodopsin2}
\end{center}
\end{figure} 



Further relaxation of the chromophore structure induces deprotonation
of Lys296's protonated Schiff base (PSB) and protonation of the
counter ion Glu113 (Fig. \ref{intro_rhodopsin2}a, e)
\cite{jaeger1994identification}. A recent computational study shows
that besides Glu113 and the chromophore, also active site water
molecules as well as Gly90 and Thr94 play significant roles during
deprotonation of the chromophore \cite{vankeulenrhodopsin}. These
results emphasise the importance of the residues in the active site as
well as the involvement of the environment on the activation pathway
of rhodopsin.

PSB deprotonation initiates further straightening of the
\emph{all}-trans retinal configuration, which induces helix
displacement as well as a change in the position of Trp265
\cite{crocker2006location} and a rearrangement of the hydrogen-bond
network around Glu122 and His211 (Fig. \ref{intro_rhodopsin2}e)
\cite{vogel2005agonists,lewis2006proton}. These signature changes that
take place on the millisecond timescale transduce the structural
rearrangements in the active site from the extracellular region to the
core of the protein. On the intracellular side, the isomerisation
ultimately affects the "closed" interaction site for G proteins that
includes the conserved residues Glu134, Arg135 and Glu247, called the
ionic lock (Fig. \ref{intro_rhodopsin2}e). As part of the activation
process, the salt bridges Glu134-Arg135 and Arg135-Glu247 are broken
and Glu134 is protonated, leading to an "open" conformation of the
intracellular region, the signalling state metarhodopsin II
(Fig. \ref{intro_rhodopsin2}c, d, e)
\cite{mahalingam2008two,choe2011crystal,deupi2012stabilized}. Rhodopsin's
meta II configuration is able to interact with G proteins in the
cytosol that will augment the external signal as one active rhodopsin
can stimulate multiple G proteins, which will ultimately lead to
vision.

  

\section{Summary and Outlook}
The current review summarizes dynamical studies on a range of systems
spanning small single molecules in the gas phase out to large proteins
in solution. Several of the systems (Mb, PDZ, rhodopsion, insulin, see
Figure \ref{fig:sketch}) can be considered as pivotal for
characterizing the intimate links between structure, dynamics and
energetics. In all of them, perturbations on short time scales (fs to
ps) lead to ensuing functional dynamics on multiple longer time
scales, spanning several orders of magnitude. Given the finding that
the type of perturbation applied to a system also imprints on its
long-time dynamical response, it is evident that characterizing the
short time dynamics is essential and mandatory for a functional
understanding of such systems which eventually should lead to
modifying and influencing it in rational ways.

\begin{figure}
\includegraphics[width=\textwidth]{fig/sketch.png}
\caption{Summary of the systems and processes studied in the present
  review.}
\label{fig:sketch}
\end{figure} 
    
One of the aims of the National Competence Center of Research on
Molecular and Ultrafast Science and Technology (MUST) is the
characterization of the short time dynamics and linking it to
structural properties and the long-time evolution of the motions at a
molecular level. The present overview should highlight that such
understanding requires a close collaboration between experimentalists
and computer-based research. A further insight is the observation that
experiments in multiple wavelength regions are required for a
comprehensive characterization of the complex interplay between
energy, motion and function. Using state-of-the art technology at all
levels, such characterizations deems to be possible.

\section*{Acknowledgments}
Financial support by the NCCR MUST is gratefully acknowledged.


\bibliography{../literature,library}

\end{document}
