\documentclass[journal=jacsat,manuscript=article]{achemso}

\usepackage{graphicx}

\bibliographystyle{achemso}

\author{Markus Meuwly}
\affiliation{Department of Chemistry, University of Basel, Klingelbergstrasse 80, CH-4056 Basel, Switzerland.}
\email{m.meuwly@unibas.ch}

\date{\today}

\title{}

\begin{document}

\begin{abstract}

\end{abstract}

\clearpage

\section{Introduction}
\label{sec:introduction} 
left empty

\section{N$_2^+$ + Ar}
\label{sec:n2plusar}
The collision of N$_2^+$ ions with Ar atoms is a prototypical system to study ion-atom collisions and charge transfer reactions. It might also be an example of a system were non-adiabatic effects are important in order to understand the underlying dynamics. Previous experiments by Schlemmer \textit{et al.}\cite{schlemmer.ijms.1999.n2plusargon} on this system have uncovered an unusual behaviour regarding the rotational relaxation of N$_2^+$ ions.
Since the (N$_2^+$-Ar) complex has a large binding energy of 1.109 eV\cite{mahnert.jcp.1995.n2plusargon}, collisions at low temperatures should proceed via a long-lived bound state. Since this usually leads to statistical mixing of all accessible product channels, a quick redistribution of rotational states can be expected. However, Schlemmer \textit{et al.} found a very low rate coefficient of $k_J$(90 K) $= (1.4\pm 0.4) \cdot 10^{-11}$ cm$^3$s$^{-1}$ for rotational relaxation, which implies that the contrary is true. Assuming that collisions occur at the Langevin rate $k_L = 7.4\cdot10^{-10}$ cm$^3$s$^{-1}$, this implies that the rotational state of N$_2^+$ ions is conserved for about 49 out of 50 collisions with Ar atoms (according to the Langevin model, a collision complex is formed if the collision energy is sufficient to overcome the rotational barrier).\cite{schlemmer.ijms.1999.n2plusargon}\\
\noindent Schlemmer \textit{et al.} proposed that this unexpected result could either be explained as the consequence of some hidden constants of motion leading to approximate selection rules, or by the presence of additional barriers in the (N$_2^+$-Ar) potential energy surface (PES). They remark that ``detailed \textit{ab initio} calculations and quantum mechanical treatment of the collision dynamics at meV energies are required to conclusively answer the question".\cite{schlemmer.ijms.1999.n2plusargon}\\

\noindent
Meuwly and coworkers\cite{unke.jcp.2016.n2plusargon,denisalpizar.2017.n2plusargon} constructed a state of the art (N$_2^+$-Ar) PES based on \textit{ab initio} energies computed at the UCCSD(T)-F12a/aug-cc-vtz level of theory using reproducing kernel Hilbert space (RKHS) theory.\cite{hollebeek.annrevphychem.1999.rkhs,unke.2017.rkhstoolkit} The RKHS method has the appealing quality that it exactly reproduces the given data, therefore allowing energy evaluations at practically \textit{ab initio} quality.\cite{unke.jcp.2016.n2plusargon} In order to assess the importance of multireference effects, additional calculations at the CASSCF/MRCI+Q level of theory were performed and it was concluded that UCCSD(T)-F12a/aug-cc-vtz is a sufficiently high level of theory.\cite{denisalpizar.2017.n2plusargon}
Quantum close-coupling calculations\cite{denisalpizar.2017.n2plusargon} as well as quasi-classical trajectory (QCT) calculations\cite{unke.jcp.2016.n2plusargon,denisalpizar.2017.n2plusargon} were performed on the RKHS-PES using conditions similar to those in the experiment by Schlemmer \textit{et al.}\cite{schlemmer.ijms.1999.n2plusargon} The rate coefficients for rotational relaxation obtained by the quantum and QCT calculations agree favourably and range between  $5.34 \cdot 10^{-10}$ cm$^3$s$^{-1}$ and $6.96 \cdot 10^{-10}$ cm$^3$s$^{-1}$.\cite{denisalpizar.2017.n2plusargon} These rates are in line with the expected quick redistribution of rotational states and suggest that more than 70\% of collisions lead to a change of the rotational quantum state.\\
\noindent Due to the high quality of the PES and the agreement of quasi-classical and quantum results, it is unlikely that the discrepancy to the experimentally observed rate is due to deficiencies of the applied computational methods. However, it is possible that the Born-Oppenheimer approximation is not valid for the (N$_2^+$-Ar) complex and non-adiabatic effects are important in the dynamics. This assumption is supported by the observation that lowest excited electronic state of  (N$_2^+$-Ar) is only energetically separated by about 700 cm$^-1$ from the ground state in the long range part of the N$_2^+$ + Ar interaction (Fig. \ref{fig:n2arlevels}). In fact, the separation of the electronic states is a function of the vibrational motion that the N$_2^+$ undergoes and therefore suggests that a coupling between nuclear and electronic degrees of freedom might exist.\cite{denisalpizar.2017.n2plusargon}

\begin{figure}
	\includegraphics[scale=0.3]{fig/n2arlevels.png}
	\caption{Interaction energy at the inner turning point ($r = 1.072 \AA$) of the $v = 0$ vibration for the three lowest electronic states of (N$_2^+$-Ar) (the angle between $r$ and $R$ is $168^{\circ}$). The curves were computed at the MRCI+Q/avtz level of theory in the $^2$A$'$ symmetry. Note that the separation between ground state and frist excited state is only about $700$ cm$^{-1}$ at around $R=4.5\AA$}
	\label{fig:n2arlevels}
\end{figure}
    
\section*{Acknowledgments}
The NCCR MUST (to MM) and the University of Basel are acknowledged.


\bibliography{../literature}

\end{document}
